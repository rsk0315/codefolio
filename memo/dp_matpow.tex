\documentclass{jsarticle}

\usepackage{rsk0315}

\newcommand{\DP}{\ensuremath\mathrm{dp}}

\begin{document}
\section{行列累乗で高速化するDP}

$\DP[i]$が$\DP[i-1]$と,ある$\DP_1[i-1]$, \dots, $\DP_k[i-1]$から計算でき,ある定数行列$A$が存在して任意の$i$について以下の式が成り立つとうれしい.
\[
\left(\begin{matrix}
  \DP[i]\\
  \DP_1[i]\\
  \vdots\\
  \DP_k[i]
\end{matrix}\right) = A \times \left(\begin{matrix}
  \DP[i-1]\\
  \DP_1[i-1]\\
  \vdots\\
  \DP_k[i-1]
\end{matrix}\right).
\]
うれしいのは以下のようにできるためである.
\[
\left(\begin{matrix}
  \DP[n]\\
  \DP_1[n]\\
  \vdots\\
  \DP_k[n]
\end{matrix}\right) = A^n \times \left(\begin{matrix}
  \DP[0]\\
  \DP_1[0]\\
  \vdots\\
  \DP_k[0]
\end{matrix}\right).
\]
行列$A$の大きさは$(k+1, k+1)$なので,乗算は$O(k^3)$で計算でき\footnote{Strassenの方法を使えば$O(k^{\log_2 7+o(1)})\subset O(k^{2.81})$で計算することもできる.},繰り返し二乗法などで$O(\log n)$回の乗算で済ませられるので,全体として$O(k^3 \log n)$時間で$\DP[n]$を計算できる.
自分でうまく$\DP_1$, \dots, $\DP_k$を決めてあげる必要があって,そこは大変そう.

\subsection{利用例}
以下のような遷移をするDPを考える.
\begin{alg}
  \caption{愚直なDP}
  {$\DP[0] \gets c$}\Comment*{初期条件は計算できるとする.}
  \For{$i \in \{1, \dots, N-1\}$}
  {
    {$\displaystyle\DP[i] \gets \sum_{j=1}^i j^2\cdot\DP[i-j]$}\;
  }
\end{alg}

\noindent
以下のように式変形をしていく.
\begin{align*}
  \DP[i]
  &= \sum_{j=1}^i j^2\cdot\DP[i-j]\\
  &= 1^2\cdot\DP[i-1] + \sum_{j=2}^i j^2\cdot\DP[i-j]\\
  &= \DP[i-1] + \sum_{j=1}^{i-1} (j+1)^2\cdot\DP[i-(j+1)]\\
  &= \DP[i-1] + \sum_{j=1}^{j-1} j^2\cdot\DP[i-1-j] + \sum_{j=1}^{i-1} 2j\cdot\DP[i-1-j] + \sum_{j=1}^{i-1} 1\cdot\DP[i-1-j]\\
  &= \DP[i-1] + \DP[i-1] + 2\cdot\sum_{j=1}^{i-1} j\cdot\DP[i-1-j] + \sum_{j=1}^{i-1} \DP[i-1-j].
\end{align*}
ここで,$\DP_1[i] = \sum_{j=1}^i j\cdot\DP[i-j]$, $\DP_2[i] = \sum_{j=1}^i \DP[i-j]$とおくと,次のようにできる.
\begin{align*}
  \DP[i] = 2\cdot\DP[i-1] + 2\cdot\DP_1[i-1] + \DP_2[i-1].
\end{align*}
$\DP_1$および$\DP_2$についても同じ手続きをする.
\begin{align*}
  \DP_1[i]
  &= \sum_{j=1}^i j\cdot\DP[i-j]\\
  &= \DP[i-1] + \sum_{j=1}^{i-1} (j+1)\cdot\DP[i-1-j]\\
  &= \DP[i-1] + \sum_{j=1}^{i-1} j\cdot\DP[i-1-j] + \sum_{j=1}^{i-1} \DP[i-1-j]\\
  &= \DP[i-1] + \DP_1[i-1] + \DP_2[i-1].\\
  \DP_2[i]
  &= \sum_{j=1}^i \DP[i-j]\\
  &= \DP[i-1] + \sum_{j=1}^{i-1} \DP[i-1-j]\\
  &= \DP[i-1] + \DP_2[i-1].
\end{align*}
これらより,次のように書けることがわかる.
\[
\left(\begin{matrix}
  \DP[i]\\
  \DP_1[i]\\
  \DP_2[i]
\end{matrix}\right) = \left(\begin{matrix}
  2 & 2 & 1\\
  1 & 1 & 1\\
  1 & 0 & 1
\end{matrix}\right) \times \left(\begin{matrix}
  \DP[i-1]\\
  \DP_1[i-1]\\
  \DP_2[i-1]
\end{matrix}\right).
\]
一般に,$\DP[i] = \sum_{j=1}^i j^k\cdot\DP[i-j]$の形のDPは上のような変形から行列累乗で解ける.
\end{document}
