\documentclass{jsarticle}

\usepackage{rsk0315}

\newcommand{\DP}{\mathrm{dp}}

\title{全方位木DP}

\begin{document}

\maketitle

\section{問題設定}
$n$頂点の木$(V, E)$と,単位元$e_\oplus$を持つモノイド$(T, \oplus)$および関数$f:T\times T\to T$に関する問題を考える.
次のように型$T$の値$\DP_r[v]$を定義する.
\[\DP_r[v] = f(\DP_r[c_1^r] \oplus \dots \oplus \DP_r[c_k^r]).\]
ここで,根を$r$としたときの頂点$v$の子を番号の昇順に$(c_1^r, \dots, c_k^r)$とする.
また,$v$が葉のときは次のように定義する.
\[\DP_r[v] = f(e_\oplus).\]
このとき,$r = \{1, \dots, n\}$について$\DP_r[r]$を求めたい.

\section{考察}

まず,以下が成り立つ.
\begin{align*}
  \DP_r[v] &= f(\DP_r[c_1^r] \oplus \dots \oplus \DP_r[c_k^r])\\
  &= f(\DP_v[c_1^r] \oplus \dots \oplus \DP_v[c_k^r]).
\end{align*}
すなわち,各$r$について$\DP_r[1], \dots, \DP_r[n]$を求める必要はなく,頂点の対$(u, v)$について$\DP_u[v]$を求める必要があるのは$u$と$v$が隣接する場合のみである.
そのため,$|V|^2$個の要素を求める必要はなく,$2|E|$個の要素を求めれば十分であることがわかる.

さて,$\DP_r[v]$を求めることを考える.
根を$r$としたときの$v$の子は$(c_1^r, \dots, c_k^r)$であった.
このとき,$v$と隣接する頂点を番号の昇順に$(u_1, \dots, u_m)$とすると$m = k+1$となる.
さらに,根を$r$としたときの$v$の親を$p_r$とすると,ある$1\le m_r\le m$が存在して$u_{m_r} = p_r$となる.
つまり,$u_1, \dots, u_m$を$m_r$を境に二分でき,以下のように書ける.
\begin{align*}
  \DP_v[c_1^r]\oplus\dots\oplus\DP_v[c_k^r]
  &= (\DP_v[u_1]\oplus\dots\oplus\DP_v[u_{m_r-1}])\oplus(\DP_v[u_{m_r+1}]\oplus\dots\oplus\DP_v[u_m])\\
  &= \left(\bigoplus_{i=1}^{u_{m_r}-1} \DP_v[c_i]\right)\oplus\left(\bigoplus_{i=u_{m_r}+1}^m\DP_v[c_i]\right)
  .
\end{align*}
これを高速に求めるためには,$\DP_v$の$\oplus$に関する左右からの累積和を持っておけばよい.
ここで,$m_r$以外は$r$に依存していないことがうれしい.

さて,各$v$について$\DP_v$の$\oplus$に左右からの累積和を求めたい.
これには二回DFSを行うとよい.
まず,頂点$1$からDFSを行うことで$1$を根とする向きの値は求められる.
すなわち,隣接する頂点対$(u, v)$について,$u$の方が$1$に近いならば$\DP_u[v]$はこのDFSで求められる.
ここで,$\DP_1$の各値は求められているので,$\DP_1$の累積和も計算できる.

その後もう一度DFSを行い,残りの$\DP_v[u]$を求める.
このとき,$\DP_v[u]$を求めるために必要な$\DP_v[??]$について,$\DP_v[u]$以外は一回目のDFSで求められている.
一方で,(探索順から)$\DP_v[u]$は$\DP_u$の累積和は求められているので,これを適切に求めることで$\DP_v[u]$も計算できる.

これらにより各$(u, v)$について$\DP_v[u]$を求めることができた.
また,$\DP_v$について累積和を求めるときに,全体の和を計算することで$\DP_v[v]$を求められ,元の問題を解くことができた.

木の辺に型$U$の重みがついているとき,$f$を$(T\times U)\times T\to T$にするとよいかも.

\vspace{10ex}
\begin{center}
  $\langle$図が入る予定.$\rangle$
\end{center}
\vspace{10ex}

\clearpage
\section{実装}

$\DP_0[v]$では$\DP_v$の左からの累積和を,$\DP_1[v]$では$\DP_v$の右からの累積和を保持する.
$\DP_0[v]$および$\DP_1[v]$は$v$の次数$\delta(v)$要素の配列である.

\subsection{擬似コード}
% 擬似コードは分割されないので順番が乱れる.float?
% そのうち上の方に図を入れるはずなので,それをしてから必要に応じて考え直す.
\begin{alg}
  \caption{全方位木DP}
  \newcommand{\PAR}{\mathrm{par}}
  \Function(\textsc{dfs0}{$(v, p)$})
  {
    {$x \gets e_\oplus$}\;
    \ForEach{neighbor $u_i$ \textbf{of} $v$}
    {
      \If{$u_i = p$}
      {
        {$\PAR[v] = i$}\Comment*{mark $i$\textsuperscript{th} neighbor as parent.}
        \Continue
      }
      {$x' \gets \text{\textsc{dfs0}}(u, v)$}\;
      {$x \gets x \oplus x'$}\;
      {$\DP_0[v][i+1] \gets \DP_1[v][i] \gets x'$}\;
    }
    \Return $f(x)$\;
  }
  \Function(\textsc{dfs1}{$(v, p, p_v)$})
  {
    \If{$v \ne 1$}
    {
      {$x \gets f(\DP_0[p][p_i] \oplus \DP_1[p][p_i+1])$}\;
      {$\DP_0[v][\PAR[v]+1] \gets \DP_1[v][\PAR[v]] \gets x$}\;
    }
    \lFor{$i \in (2, \dots, \delta(v))$}
      {$\DP_0[v][i] \gets \DP_0[v][i-1] \oplus \DP_0[v][i]$}
    \lFor{$i \in (\delta(v)-1, \dots, 1)$}
      {$\DP_1[v][i] \gets \DP_1[v][i] \oplus \DP_1[v][i+1]$}
    {$\DP[v] \gets \DP_1[v][0]$}\;
    \ForEach{neighbor $u_i$ \textbf{of} $v$}
    {
      \lIf{$u_i \ne p$} {$\text{\textsc{dfs1}}(u_i, v, i)$}
    }
  }
  {$\text{\textsc{dfs0}}(1, 0), \text{\textsc{dfs1}}(1, 0, 0)$}\;
\end{alg}

\subsection{C++}
\begin{minted}{c++}
  template <typename Monoid, typename UndirectedTree>
  auto dp_on_tree(UndirectedTree const& g) {
    Monoid e{};
    size_t n = g.size();
    std::vector<size_t> parent(n, -1_zu);
  
    std::vector<std::vector<Monoid>> dp0(n), dp1(n);
    std::vector<Monoid> dp(n);
    for (size_t i = 0; i < n; ++i) {
      dp0[i].resize(g[i].size()+1, e);
      dp1[i].resize(g[i].size()+1, e);
    }
  
    make_fix_point([&](auto dfs0, size_t v, size_t p) -> Monoid {
        Monoid res = e;
        typename UndirectedTree::weight_type weight{};
        for (size_t i = 0; i < g[v].size(); ++i) {
          size_t u = g[v][i].target();
          if (u == p) {
            parent[v] = i;
            weight = g[v][i].weight();
            continue;
          }
          Monoid tmp = dfs0(u, v);
          res.append(tmp);
          dp0[v][i+1] = dp1[v][i] = tmp;
        }
        return res.f(weight);
    })(0, -1_zu);
  
    make_fix_point([&](auto dfs1, size_t v, size_t p, size_t pi) -> void {
        if (v != 0) {
          Monoid tmp = (dp0[p][pi] + dp1[p][pi+1]).f(g[p][pi].weight());
          dp0[v][parent[v]+1] = tmp;
          dp1[v][parent[v]] = tmp;
        }
        {
          for (size_t i = 1; i < dp0[v].size(); ++i)
            dp0[v][i].prepend(dp0[v][i-1]);
          for (size_t i = dp1[v].size()-1; i--;)
            dp1[v][i].append(dp1[v][i+1]);
          dp[v] = dp1[v][0];
        }
        for (size_t i = 0; i < g[v].size(); ++i) {
          size_t u = g[v][i].target();
          if (u != p) dfs1(u, v, i);
        }
    })(0, -1_zu, -1_zu);
  
    return dp;
  }
\end{minted}

\end{document}
