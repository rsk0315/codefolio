\documentclass{jsarticle}

\usepackage{rsk0315}

\newcommand{\floor}[1]{\lfloor #1\rfloor}
\newcommand{\ceil}[1]{\lceil #1\rceil}

\title{Mo's Algorithm}

\begin{document}
\maketitle

静的な配列に対して,区間$[L_i, R_i)$に関するクエリ処理を行う.
\iffalse
]\fi
ある区間$[s, t)$についての結果が求められているときに$[s\pm1, t\pm1)$に関するクエリに$O(T(n))$で求められるとする.
\iffalse
]]\fi

配列を個数$b$のバケットで分割する.バケットの個数は$\ceil{n/b}$である.
クエリの区間$[L_i, R_i)$を$(\floor{L_i/b}, R_i)$をキーとしてソートし,順に処理することを考える.
\iffalse
]\fi

まず,区間の左端がどの程度変更されるかを考える.

う

左端の属するバケットが同じとき,左端の移動は at most $b$ per query.右端の移動は at most $n$.
左端の属するバケットが次のものに変わるとき,左端の移動は at most $2b$.右端の移動は at most $n$.バケットが変わるのは at most $\ceil{n/b}-1$ times.

これより,左端の移動は$bq + 2b(\ceil{n/b}-1) = bq+O(n)$で,右端の移動は$2n(\ceil{n/b}-1) = O(n^2/b)$.
よって,$bq\cdot T(n)$と$n^2/b\cdot T(n)$をバランスよくするような$b$を定めたい.
相加・相乗平均の関係から,以下が成り立つ.
\begin{align*}
  (bq+n^2/b)\cdot T(n)
  &\ge \big((bq\cdot T(n))\cdot(n^2/b\cdot T(n))\big)^{1/2}\\
  &= n\sqrt q\cdot T(n)\\
\end{align*}
等号が成り立つのは$bq=n^2/b$のときで,$b=n/\sqrt{q}$程度に選ぶのがよさそう.

\end{document}
