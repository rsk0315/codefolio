\documentclass{jsarticle}

\usepackage{rsk0315}
\usepackage{amsthm}

\newcommand{\DP}{\ensuremath\mathrm{dp}}

\begin{document}
\section{包除原理}

数え上げのテクニックの一つ.
集合$A$が与えられ,その要素を変数とする述語たち$\mathcal{P}=\{P_1, \dots, P_n\}$を考える.
$1\le i\le n$に対して$A$の部分集合$A_i$を$A_i=\{a\in A\mid P_i(a)\}$で定めるとき,$\left|\bigcup_{i=1}^n A_i\right|$を求めるものである.
$\Lambda_n=\{1, \dots, n\}$とする.
\begin{align}\label{eqn:base}
\left|\bigcup_{i=1}^n A_i\right|
&= \sum_{\emptyset\subset\Lambda\subseteq \Lambda_n} (-1)^{|\Lambda|-1}\cdot \left|\bigcap_{i\in\Lambda} A_i\right|.\\
\label{eqn:base2}
&= \sum_{j=1}^n (-1)^{j-1}\cdot \left(\sum_{|\Lambda|=j} \left|\bigcap_{i\in\Lambda} A_i\right|\right).
\end{align}
$\Lambda$の要素数を固定したときに$\left|\bigcap_{i\in\Lambda} A_i\right|$の総和をDPなどで求められるなら,式(\ref{eqn:base2})を用いて計算することができる.また,特に
$|\Lambda|=|\Lambda'|\implies\left|\bigcap_{i\in\Lambda}A_i\right| = \left|\bigcap_{i\in\Lambda'}A_i\right|$であるなら,$|\Lambda|=j$であるような$\Lambda$の代表元を$\Lambda^j$と書くことにして次のように変形できる.
\begin{align}\label{eqn:linear}
  \left|\bigcup_{i=1}^n A_i\right|
  = \sum_{j=1}^n (-1)^{j-1}\cdot {}_n C_j\cdot\left|\bigcap_{i\in\Lambda^j}A_i\right|.
\end{align}
式(\ref{eqn:base})においては$\sum$で足される項が$2^n-1$個だったのに対し,式(\ref{eqn:base2})--(\ref{eqn:linear})では$n$に減っていてうれしい.

上の議論は,$\left|\bigcap_{i\in\Lambda}A_i\right|$を計算するのが容易であることを前提としているが,逆に$\left|\bigcap_{i\in\Lambda}(A\setminus A_i)\right|$の計算が容易な状況\footnote{満たす条件を決め打ちするよりも,満たさない条件を決め打ちした方が楽な場合.}
で$\left|\bigcap_{i=1}^n A_i\right|$を求めたいときには以下のようにするとよい.
\[
\bigcap_{i=1}^n A_i = A\setminus\left(\bigcup_{i=1}^n (A\setminus A_i)\right).
\]
すなわち,
\begin{align}\label{eqn:cmpl}
  \left|\bigcap_{i=1}^n A_i\right| = |A|-\left|\bigcup_{i=1}^n (A\setminus A_i)\right|.
\end{align}
右辺の第2項は,$A\setminus A_i$を$A_i$と置き直すことで式(\ref{eqn:base})の枠組みで求められる.

\subsection{証明}
$\mathcal{P}$の大きさ$n$に関する帰納法で示す.$n=2$の場合はVenn図などから示される(省略).
$n<k$で成り立っていることを仮定する.
まず,$n=2$の場合の式を用いて次のように変形する.
\begin{align*}
  \left|\bigcup_{i=1}^k A_i\right| &= \left|\left(\bigcup_{i=1}^{k-1} A_i\right)\cup A_k\right|\\
  &= \left|\bigcup_{i=1}^{k-1} A_i\right| + |A_k| - \left|\left(\bigcup_{i=1}^{k-1} A_i\right)\cap A_k\right|\\
  &= \left|\bigcup_{i=1}^{k-1} A_i\right| + |A_k| - \left|\bigcup_{i=1}^{k-1} (A_i\cap A_k)\right|\\
\end{align*}
さらに,$n=k-1$の場合の式を用いて右辺の各項は次のように変形できる.
\begin{align*}
  \left|\bigcup_{i=1}^{k-1} A_i\right|
  &= \sum_{\emptyset\subset\Lambda\subseteq\Lambda_{k-1}} (-1)^{|\Lambda|-1} \cdot\left|\bigcap_{i\in\Lambda} A_i\right|.
\end{align*}
\begin{align*}
  |A_k| = \sum_{\Lambda=\{k\}} (-1)^{1-1} \cdot\left|\bigcap_{i\in\Lambda} A_i\right|.
\end{align*}
\begin{align*}
  -\left|\bigcup_{i=1}^{k-1} (A_i\cap A_k)\right|
  &= -\sum_{\emptyset\subset\Lambda\subseteq\Lambda_{k-1}} (-1)^{|\Lambda|-1} \cdot\left|\bigcap_{i\in\Lambda} (A_i\cap A_k)\right|\\
  &= -\sum_{\emptyset\subset\Lambda\subseteq\Lambda_{k-1}} (-1)^{|\Lambda|-1} \cdot\left|\left(\bigcap_{i\in\Lambda} A_i\right)\cap A_k\right|\\
  &= \sum_{\emptyset\subset\Lambda\subseteq\Lambda_{k-1}} (-1)^{|\Lambda\cup\{k\}|-1} \cdot\left|\bigcap_{i\in\Lambda\cup\{k\}} A_i\right|.
\end{align*}
第1項は$A_k$を含まない$k-1$個以下の積集合,第2項は$A_k$のみからなる集合,第3項は$A_k$を含む$2$個以上$k$個以下の積集合に関する式になっており,次のようにまとめることができる.
\begin{align*}
  \left|\bigcup_{i=1}^k A_i\right|
  &= \sum_{\emptyset\subset\Lambda\subseteq\Lambda_k} (-1)^{|\Lambda|-1} \cdot\left|\bigcap_{i\in\Lambda} A_i\right|.\qed
\end{align*}

\subsection{発展}
\subsubsection{高速ゼータ変換・高速M\"obius変換}
高速ゼータ変換は,集合$S$の部分集合を引数に取る関数$f$に対して$g(S)=\sum_{T\subseteq S} f(T)$を$O(|S|\cdot 2^{|S|})$時間で求める.
高速M\"obius変換はその逆変換に相当し,$f(S)=\sum_{T\subseteq S} (-1)^{|S\setminus T|}\cdot g(T)$を求める.
\begin{minted}{c++}
  // fast zeta transformation
  for (size_type i = 0; i < n; ++i)
    for (size_type j = 0; j < (1_zu << n); ++j)
      if (j >> i & 1_zu) dp[j] += dp[j ^ (1_zu << i)];
\end{minted}
\begin{minted}{c++}
  // fast Moebius transformation
  for (size_type i = 0; i < n; ++i)
    for (size_type j = 0; j < (1_zu << n); ++j)
      if (j >> i & 1_zu) dp[j] -= dp[j ^ (1_zu << i)];
\end{minted}
各\texttt{j}について$\mathtt{dp}[\mathtt{j}] = f(\mathtt{j})$で初期化し,ゼータ変換を施すと$\mathtt{dp}[\mathtt{j}] = g(\mathtt{j})$になっている.M\"obius変換についても$g$と$f$が逆になること以外は同様である.
ここで,\texttt{j}は部分集合を2進数でエンコードしたものである.

たとえば,$\Lambda_n$の各部分集合$S$に対して式(\ref{eqn:base})の包除原理を行うことを考える.すなわち,各$S$に対して$f(S)=\sum_{T\subseteq S} (-1)^{|T|-1}\cdot \left|\bigcap_{i\in T}A_i\right|$を計算する.
これは,$g(\emptyset)=0$,$g(T) = \left|\bigcap_{i\in T} A_i\right|$として高速M\"obius変換で得られる$f(S)$を用いて$(-1)^{|S|}\cdot f(S)$として計算できる.
これにより,愚直には$O(3^{|S|})$かかる計算を$O(|S|\cdot 2^{|S|})$で行うことができる\footnote{目安としては,前者が$n\le 15$で後者が$n\le 20$程度まで許容できる.}.

\subsubsection{DP}
式(\ref{eqn:base2})より,$j$個の条件を満たす場合の数を求めることを考える.
式(\ref{eqn:cmpl})を用いて$j$個の条件に違反する場合の数を考えることもできる.

$\DP[i][j]$では,$i$番目までの条件を見て,そのうち$j$個の条件を満たす場合の数(を求めるための値\footnote{遷移のしやすさと相談するとよさそう.})として定義する.
$\DP[i][j]$から$\DP[i+1][j+1]$への遷移($i$番目の条件を満たす)と$\DP[i+1][j]$への遷移($i$番目の条件を考慮しない)をそれぞれがんばって考える.
その後,$\DP[n][j]$から$\sum_{|\Lambda|=j}\left|\bigcap_{i\in\Lambda} A_i\right|$を求めることで式(\ref{eqn:base2})を計算できる.
$\DP[n][j]$が直接その値を表す場合,(遷移によっては)$j$を指数部の偶奇を示す$\{0, 1\}$のみにまとめることができうる.

また,式(\ref{eqn:cmpl})を用いる場合,$\DP[n][0]$は「$n$個目までの条件を見て$0$個の条件を考慮する場合」であり,これは集合全体に対応している.すなわち,以下のように簡単にできる(はず).$f$は適当に定義する.
\begin{align*}
  \left|\bigcap_{i=1}^n A_i\right|
  &= |A| - \sum_{j=1}^n (-1)^{j-1}\cdot \left(\sum_{|\Lambda|=j} \left|\bigcap_{i\in\Lambda} (A\setminus A_i)\right|\right)\\
  &= f(0, \DP[n][0]) - \sum_{j=1}^n (-1)^{j-1}\cdot f(j, \DP[n][j])\\
  &= \sum_{j=0}^n (-1)^j\cdot f(j, \DP[n][j]).
\end{align*}

状態数が$O(n^2)$になる場合,$n\le 5000$などではMLEしうるので,一次元配列を使いまわすテクを使うとよい.
また,操作回数など別のパラメータが必要になる場合は適宜DPの次元を増やす必要がある(それはそう).

\subsubsection{約数系包除}
ある$n$について$f(n)$を求めた後,$g(n) = f(n) - \sum_{n'\sqsubset n} g(n')$を求める\footnote{$n'\sqsubseteq n$は$n'$が$n$の約数であることを表し,$n'\sqsubset n$は$n'$が$n$の真の約数である,つまり$n'\sqsubset n\iff n'\sqsubseteq n\wedge n'\neq n$であることを表す.}局面が考えられる.
典型的な例では,周期が関連する数え上げなどが挙げられる.
より具体的には,たとえば周期$4$の数列には周期$2$や$1$の数列も含まれるので,最小周期が$4$の数列を数え上げたい場合にはそれらを取り除く必要がある.

さて,ここで以下の関係式が成り立つ(M\"obiusの反転公式).
\[f(n) = \sum_{n'\sqsubseteq n} g(n') \iff g(n) = \sum_{n'\sqsubseteq n} \mu(n/n')\cdot f(n').\]
ここで,$\mu(n)$は以下で定義されるM\"obius関数である\footnote{$g(n)$を$f(n')$の足し引きで求める際に,足されるのか引かれるのか,あるいは打ち消されるのかを表す関数である.}.
\[
\mu(n) = \begin{cases}
  1 & \text{if }n = 1,\\
  0 & \text{if }{}^\exists m > 1\text{ s.t. }m^2\sqsubseteq n,\\
  (-1)^k & \text{if }n = \prod_{i=1}^k p_i\text{, where each }p_i\text{ is prime}.
\end{cases}
\]
$n'\sqsubseteq n$である各$n'$に対する$\mu(n')$は,$n$の素因数の列挙をしておくとbit DPで求めることができる.
また,$n$の上限によってはEratosthenesの篩を応用して$1\le n'\le n$に対する$\mu(n')$を計算しておくこともできる.

%% \subsubsection{その他}
%% Everything on It的なのを書く.
%% 的なのってのはどの要素を指してるんですかね...
%% べき集合に関するのを書こうとしたけど,本質的にはそこまで変わらなくない?
%% 考えが変わったらどうにかします

\subsection{補足}
$\mathcal{P}$の各部分集合$\mathcal{P}'$について$\left|\bigcap_{i\in\mathcal{P}'} A_i\right|$を前計算しておくことで,各クエリで与えられる$\mathcal{Q}$について$\left|\bigcup_{i\in\mathcal{Q}} A_i\right|$を高速に求められたりする.
% Uncommon

%\subsubsection{暗黙の制約}
$n\le 10^9$,$n\le 10^{18}$なら$n$の持つ素因数の個数はそれぞれ,たかだか$9$個,$15$個なので,$n$が多少大きくても$n$の素因数に関する部分集合であれば式(\ref{eqn:base})をそのまま計算することができる.

\end{document}
