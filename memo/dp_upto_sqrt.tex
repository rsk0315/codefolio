\documentclass{jsarticle}

\usepackage{rsk0315}

\newcommand{\DP}{\ensuremath\mathrm{dp}}

\begin{document}
\section{状態を$O(\sqrt n)$個にまとめるDP}

愚直には以下のような遷移をするDPを考える.
\begin{alg}
  \caption{愚直なDP}
  \For{$i \in \{1, \dots, K\}$}
  {
    \For{$j \in \{1, \dots, N\}$}
    {
      {$\displaystyle\DP[i][j] \gets \sum_{k=1}^{\lfloor N/j\rfloor} \DP[i-1][k]$}\;
    }
  }
\end{alg}

\noindent
$\sum$の部分は累積和を用意することで$O(1)$で計算可能としても,全体では$O(KN)$時間と$O(N)$要素ぶんの空間が必要となる\footnote{前のテーブルと今のテーブルだけ確保するやつを必要に応じて使う.}.

ここで,$\sum$の上限である$\lfloor N/j\rfloor$が等しい$j$たちについては$\DP[i][j]$の値は同じになることがわかる.
$j < \sqrt N$である$j$は$O(\sqrt N)$個しかなく,$j \ge \sqrt N$である$j$に対して$\lfloor N/j\rfloor$は$O(\sqrt N)$個しかないことから,状態の個数を$O(\sqrt N)$個にまとめられることがわかる.
また,以下のようにしてその状態たちを求めることができる.
\begin{alg}
  \caption{まとめる状態を求める}
  {$S \gets \{0\}$}\;
  {$i \gets 1$}\;
  \While{$i\times i \le N$}
  {
    {$S \gets S \cup \{i, \lfloor N/i\rfloor\}$}\;
    {$i \gets i+1$}\;
  }
\end{alg}
\noindent
この処理のあと
\[S = \{a_0, a_1, \dots, a_{M-1}\},\text{ where }a_0 = 0 < a_1 < \dots < a_{M-1}\]
とすると,各$j$について$k \in (a_{j-1}, a_j]$は同じ状態としてまとめられる.
まとめられた状態から別の状態に値を渡すとき,まとめた個数を掛ける必要があるので,次のようになる.
以下で,$\DP[i][j]$は各$i$に関して$(a_{j-1}, a_j]$でまとめた状態の値を管理する.
時間計算量は$O(KM) = O(K\sqrt N)$.
\begin{alg}
  \caption{状態をまとめたあとのDP}
  \For{$i \in \{1, \dots, K\}$}
  {
    \For{$j \in \{1, \dots, M-1\}$}
    {
      {$\displaystyle\DP[i][j] \gets \left(\sum_{k=1}^{M-j}\DP[i-1][k]\right) \times (a_j-a_{j-1})$}\;
    }
  }
\end{alg}

\begin{minted}{c++}
  int main() {
    size_t N, K;
    scanf("%zu %zu", &N, &K);
  
    std::vector<size_t> a{0};
    for (size_t i = 1; i * i <= N; ++i) {
      a.push_back(i);
      a.push_back(N/i);
    }
    std::sort(a.begin(), a.end());
    a.erase(std::unique(a.begin(), a.end()), a.end());
  
    size_t M = a.size();
    std::vector<std::vector<intmax_t>> dp(K, std::vector<intmax_t>(M));
    for (size_t j = 1; j < M; ++j) {
      dp[0][j] = a[j] - a[j-1];
      (dp[0][j] += dp[0][j-1]) %= mod;
    }
  
    for (size_t i = 1; i < K; ++i) {
      for (size_t j = 1; j < M; ++j) {
        dp[i][j] = dp[i-1][M-j] * (a[j]-a[j-1]) % mod;
        (dp[i][j] += dp[i][j-1]) %= mod;
      }
    }
  
    printf("%jd\n", dp[K-1][M-1]);
  }
\end{minted}

\end{document}
