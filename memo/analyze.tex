\documentclass{jsarticle}

\usepackage{rsk0315}

\newcommand{\floor}[1]{\left\lfloor #1\right\rfloor}
\newcommand{\ceil}[1]{\left\lceil #1\right\rceil}

\title{計算量解析に関する自明でない例}

\begin{document}
\maketitle

\section{調和級数}

以下のようなループの回数を解析する.
\begin{minted}{c++}
  for (size_type i = 0; i < n; ++i)
    for (size_type j = 0; j < n; i += j)
      ...
\end{minted}
$i$を固定したとき,内側のループは$\floor{n/i}$回なので,全体は以下のようになる.
\begin{align*}
  \floor{\frac{n}{1}}+\floor{\frac{n}{2}}+\dots+\floor{\frac{n}{n}}
  &\le \sum_{i=1}^n \frac{n}{i}\\
  &= n\cdot\sum_{i=1}^n \frac{1}{i}\\
  &= n\cdot(\log_e n+\gamma+\varepsilon_n)\\
  &\subseteq n\cdot(\log_e n+\gamma+o(1))\\
  &\subseteq n\log_e n+\Theta(n)\\
  &\subseteq \Theta(n\log n).
\end{align*}
定数倍を細かく見ることはなさそうなので,$\Theta(n\log n)$回と見ることが多そう.

\section{最小値を選ぶやつ}

これはたぶん特殊な例.適宜一般化する.

$\Sigma$上の長さ$n$の文字列を考える.
$n_\sigma$を文字$\sigma$の出現数としたとき,以下の式を解析する\footnote{二種類のアルゴリズムのうち,効率的な方を選ぶ場合の計算量解析など.}.
\[
\sum_{\sigma\in\Sigma} \min(n_\sigma^2, n\log n).
\]
これが最大となるのは$n_\sigma^2 = n\log n$のときのはずで,以下のようになる.
\begin{align*}
  &\phantom{{}={}}\sum_{\sigma\in\Sigma} \min(n_\sigma^2, n\log n)\\
  &= \sum_{\sigma\in\Sigma}\sqrt{n_\sigma^2}\cdot\sqrt{n\log n}\\
  &= \left(\sum_{\sigma\in\Sigma}n_\sigma\right)\cdot\sqrt{n\log n}\\
  &= n\cdot\sqrt{n\log n}.
\end{align*}

$\min$の引数のうち,片方が$n$に依存していて,他方が束縛変数に依存している場合に,それらが等しいとおいて変形する? 束縛変数のみの式にして,閉じた形にできるとうれしい.

\section{二乗の木DP}

二分木での木DPを考える.頂点$v_p$と,その子$v_l$, $v_r$があり,それらの部分木のサイズがそれぞれ$n_p$, $n_l$, $n_r$であるときの計算量が以下のようになっているとする.
\[T(n_p) = T(n_l)+T(n_r)+O(lr).\]
全体の頂点数が$n$のとき,これは$T(n) = O(n^2)$となる.

多分木なら$O(lr)$の項はどう書くとよい? あるいは(二分木でも)その項がもっと大きかったら?

\end{document}
