\documentclass{jsarticle}

\usepackage{rsk0315}

\newcommand{\DP}{\ensuremath\mathrm{dp}}

\begin{document}
\section{全方位木DP}
$n$頂点の木が与えられる.ある頂点$r$を根として以下のようなDPを考える.
\begin{itemize}
\item 頂点$i$が葉のとき$\DP[i]$は既知である.
\item 頂点$i$が内部節点で$(c_{i,1}, \dots, c_{i,m})$を子に持つとき$\DP[i]$は$(\DP[c_{i,1}], \dots, \DP[c_{i,m}])$から計算できる.
\item 最終的に求めたい答えは$\DP[r]$である.
\end{itemize}
このアルゴリズムは,上のDPで$r=1, \dots, n$としたときそれぞれについての$\DP[r]$を$O(n)$で求める\footnote{ここ正確ではなくて,各マージの際の単位演算を$O(n)$回行うみたいなことが言いたい.}.

いま,適当な関数$f$と演算$\oplus$が定義されて$\DP[i]$が次のように計算できるとする.
\[\DP[i] = f(\DP[c_{i,1}]) \oplus \dots \oplus f(\DP[c_{i,m}])\]
頂点$i$を根として見たとき,$i$のある子$c_{i,k}$以下の部分木がない木における$\DP[i]$を$\DP[i]_k$と書くことにすると,これは次のようになっているはずである.
\[\DP[i]_k = f(\DP[c_{i,1}]) \oplus \dots \oplus f(\DP[c_{i,k-1}]) \oplus f(\DP[c_{i,k+1}]) \oplus \dots \oplus f(\DP[c_{i,m}])\]
このアルゴリズムではこれを高速に求められる必要があるので,左右からの累積和を用いて次のように書けるとする\footnote{適当に単位元を番兵としておいておく.たぶん葉での値が単位元になると思う.}.
\begin{align*}
  \DP_L[i][j] &= f(\DP[c_{i,1}]) \oplus \dots \oplus f(\DP[c_{i,j-1}])\\
  \DP_R[i][j] &= f(\DP[c_{i,j}]) \oplus \dots \oplus f(\DP[c_{i,m}])\\
  \DP[i]_k &= \DP_L[i][k] \oplus \DP_R[i][k+1]
\end{align*}

全体としては,DFSを2回行う.
最初のDFSでは頂点$1$を根としたときのDPを求め,次のDFSではその結果を元に各頂点を根としたときのDPを求める.

木を隣接リスト表現で管理しているとする.このとき,$i$に関する隣接リストの$j$番目の要素は,対象の木を$i$を根として見たときの$j$番目の子として考えることができる.
また,頂点$i$に関する隣接リストの$j$番目の要素が,頂点$1$を根としたときの$i$の親であるとき,$p[i] = j$とする.

各頂点$i$に対し,$i$の次数が$\delta(i)$のとき,$\DP_L[i]$および$\DP_R[i]$を要素数$\delta(i)+1$の配列としておき,各要素を$\oplus$の単位元で初期化しておく.
$\DP[i]$を$i$を根としたときの問題の答えとする.

\newpage
\begin{alg}
  \caption{最初のDFS}
  \Function(\textsc{Dfs0}{$(i, p)$})
  {
    {$x \gets d$}\Comment*{答えをDPの初期値で初期化}
    \For{$j \in \{1, \dots, n\}$}
    {
      {$u \gets g[i][j]$}\;
      \If{$u = p$}{
        {$p[i] \gets j$}\;
        \Continue\;
      }
      {$y \gets f(\text{\textsc{Dfs0}}(u, i))$}\;
      {$x \gets x \oplus y$}\;
      {$\DP_L[i][j+1] \gets \DP_R[i][j] \gets y$}\;
    }
    \Return{$x$}\;
  }
\end{alg}

\begin{alg}
  \caption{二回目のDFS}
  \Function(\textsc{Dfs1}{$(i, p, p_i)$})
  {
    \If{$i \neq 1$}
    {
      {$y \gets f(d \oplus \DP_L[p][p_i] \oplus \DP_R[p][p_i+1])$}\;
      {$\DP_L[i][p[i]+1] \gets \DP_R[i][p[i]] \gets y$}\;
    }
    \For{$j \in \{2, \dots, \delta(i)\}$}
    {
      {$\DP_L[i][j] \gets \DP_L[i][j] \oplus \DP_L[i][j-1]$}\;
    }
    \For{$j \in \{\delta(i)-1, \dots, 1\}$}
    {
      {$\DP_R[i][j] \gets \DP_R[i][j] \oplus \DP_R[i][j+1]$}\;
    }
    {$\DP[i] = (d \oplus \DP_R[i][1])$}\;
    \For{$j \in \{1, \dots, \delta(i)\}$}
    {
      {$u \gets g[i][j]$}\;
      \lIf{$i \neq p$}{$\text{\textsc{Dfs1}}(u, i, j)$}
    }
  }
\end{alg}

$\text{\textsc{Dfs0}}(1, 0)$および$\text{\textsc{Dfs1}}(1, 0, 0)$を呼び出すとよい.

\begin{minted}{c++}
  std::vector<intmax_t> dp_on_tree(const tree& g) {
    size_t n = g.size();
    std::vector<size_t> parent(n, -1);
  
    std::pair<intmax_t, intmax_t> const leaf(1, 1);
    std::pair<intmax_t, intmax_t> const unit(0, 0);
    auto f = [](std::pair<intmax_t, intmax_t> const& c) {
        return std::make_pair(c.first, c.first+c.second);
    };
  
    std::vector<std::vector<std::pair<intmax_t, intmax_t>>> dp0(n), dp1(n);
    std::vector<intmax_t> dp(n);
    for (size_t i = 0; i < n; ++i) {
      dp0[i].resize(g[i].size()+1, unit);
      dp1[i].resize(g[i].size()+1, unit);
    }
  
    make_fix_point([&](auto dfs0, size_t v, size_t p) -> std::pair<intmax_t, intmax_t> {
        std::pair<intmax_t, intmax_t> res = leaf;
        for (size_t i = 0; i < g[v].size(); ++i) {
          size_t u = g[v][i];
          if (u == p) {
            parent[v] = i;
            continue;
          }
          std::pair<intmax_t, intmax_t> tmp = f(dfs0(u, v));
          res += tmp;
          dp0[v][i+1] = dp1[v][i] = tmp;
        }
        return res;
    })(0, -1);
  
    make_fix_point([&](auto dfs1, size_t v, size_t p, size_t pi) -> void {
        if (v != 0) {
          std::pair<intmax_t, intmax_t> tmp = f(leaf + dp0[p][pi] + dp1[p][pi+1]);
          dp0[v][parent[v]+1] = tmp;
          dp1[v][parent[v]] = tmp;
        }
        {
          for (size_t i = 1; i < dp0[v].size(); ++i)
            dp0[v][i] += dp0[v][i-1];
          for (size_t i = dp1[v].size()-1; i--;)
            dp1[v][i] += dp1[v][i+1];
    
          dp[v] = (leaf + dp0[v][0] + dp1[v][0]).second;
        }
        for (size_t i = 0; i < g[v].size(); ++i) {
          size_t u = g[v][i];
          if (u != p) dfs1(u, v, i);
        }
    })(0, -1, -1);
  
    return dp;
  }
\end{minted}

\end{document}
