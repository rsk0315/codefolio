\documentclass{jsarticle}

\usepackage{rsk0315}

\usepackage{mathtools}

\begin{document}
\section{線形計画問題}

線形関数の制約で書ける最適化問題を線形計画問題と呼ぶ.
例を以下に挙げる:
\begin{alignat*}{2}
  & \text{minimize } & & -2x_1 + 3x_2 \\
  & \text{subject to }& \quad & \begin{aligned}[t]
    \phantom{1}x_1 + \phantom{1}x_2 &= 7,\\
    \phantom{1}x_1 - 2x_2 &\le 4,\\
    \phantom{1}x_1 \phantom{{} + 0x_2} &\ge 0.
  \end{aligned}
\end{alignat*}

% https://tex.stackexchange.com/questions/184937/align-a-linear-program
%% \begin{alignat*}{2}
%%   & \text{minimize: } & & \sum_{j=1}^{m} w_{j}x_{j} \\
%%    & \text{subject to: }& \quad & \sum_{\mathclap{{j:e_{i} \in S_{j}}}}\begin{aligned}[t]
%%                     x_{j} & \geq 1,& i & =1, \dots, n\\[3ex]
%%                   x_{j} & \in \{0,1\}, & \quad j &=1 ,\dots, m
%%                 \end{aligned}
%% \end{alignat*}

この問題を解く方法を線形計画法といい,今回はその一つとしてsimplex法を紹介する.

\subsection{制約の形式}
最初に,standard form / slack formと呼ばれる二つの形式について述べる.

\subsubsection{standard form}
standard formは以下のような形式を持つ.
\begin{enumerate}
\item 最大化の形である.
  \begin{itemize}
  \item 線形関数のminimizeではなくmaximizeをする.
  \item minimize $\sum_j c_jx_j$をmaximize $\sum_j -c_jx_j$に書き換えると変換可能.
  \end{itemize}
\item 各変数$x_j$について非負制約$x_j \ge 0$がある.
  \begin{itemize}
  \item 非負制約のない$x_j$を$x_j = x_j'-x_j''$で置き換え,これらに非負制約をつけることで変換可能.
  \end{itemize}
\item 等号制約ではなく不等号制約の形である.
  \begin{itemize}
  \item $\sum_j a_{ij}x_j = b_i$を$\sum_j a_{ij}x_j \le b_i$と$\sum_j a_{ij}x_j \ge b_i$の二つの条件に書き換えると変換可能.
  \end{itemize}
\item 不等式制約は$\ge$ではなく$\le$の形の制約である.
  \begin{itemize}
  \item これは各変数の非負制約については除外されるはず.
  \item $\sum_j a_{ij}x_j \ge b_i$を$\sum_j -a_{ij}x_j \le -b_i$に書き換えると変換可能.
  \end{itemize}
\end{enumerate}

すなわち,standard formは$m\times n$行列$\bm{A} = (a_{ij})$,$m$次元ベクトル$\bm{b} = (b_i)$,$n$次元ベクトル$\bm{c} = (c_j)$, $\bm{x} = (x_j)$を用いて以下のように表せる.
\begin{alignat*}{2}
  & \text{maximize } & & \bm{c}^\top\bm{x} \\
  & \text{subject to }& \quad & \begin{aligned}[t]
    \bm{A}\bm{x} \le \bm{b}, \\
    \phantom{\bm{A}}\bm{x} \ge \bm{0}.
  \end{aligned}
\end{alignat*}
ここで,ベクトルについての不等式は,各要素に対してその不等式が成り立つことを意味する.

\subsubsection{slack form}
次に,slack formについて述べる.
standard formにおける次の式がある.
\[\sum_{j=1}^n a_{ij}x_j \le b_i.\]
これを次の二つの制約に書き換える.
この書き換えによって得られる形式がslack formである.
\begin{align*}
  x_{n+i} &= b_i - \sum_{j=1}^n a_{ij}x_j,\\
  x_{n+i} &\ge 0.
\end{align*}
slack formにおいて,左辺に現れる変数たち$x_{n+i}$をbasic variablesといい,右辺に現れる変数たち$x_j$をnonbasic variablesという.

\subsection{simplex法}
線形計画問題をslack formの形式で扱い,大まかには以下のような処理をする.
あるbasicな変数とnonbasicな変数を一つずつ選び,それらを交換する\footnote{basicな変数をnonbasicになるようにし,nonbasicな変数をbasicになるように書き換える.}ことを繰り返しながら最大化を行う.

\end{document}
