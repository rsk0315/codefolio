\documentclass{jsarticle}

\usepackage{rsk0315}

\newcommand{\DP}{\ensuremath\mathrm{dp}}

\begin{document}
\section{CHTで高速化するDP}

以下のような遷移をするDPを考える.
\begin{alg}
  \caption{愚直なDP}
  {$\DP[0] \gets c$}\Comment*{初期条件は計算できるとする.}
  \For{$i \in \{1, \dots, N-1\}$}
  {
    {$\displaystyle\DP[i] \gets \min_{0\le j < i}\,\{p(j)\cdot q(i) + r(j)\} + s(i)$}\;
  }
\end{alg}

\noindent
$p(j)$や$r(j)$は$\DP[j]$を含む式でもいいし,関係ない式でも問題ない.

ここで,直線の集合に関する以下の処理をできるデータ構造を用意する.これはconvex hull trickなどと呼ばれるものである.
\begin{itemize}
\item 直線$y=ax+b$を追加する
\item 管理している直線のうち,$x=x_0$での$y$の最小値を答える
\end{itemize}

これを用いて上のDPを以下のように高速化できる.
直線の集合$\mathcal{S}$をこのデータ構造を用いて管理する.
% また,$\DP[0]$の値は計算できているとする.
\begin{alg}
  \caption{CHTで高速化したDP}
  %% {$\mathcal{S} \gets \{p(0)\cdot x+r(0)\}$}\;
  {$\DP[0] \gets c$}\;
  {$\mathcal{S} \gets \{\}$}\;
  \For{$i \in \{1, \dots, N-1\}$}
  {
    {$\mathcal{S} \gets \mathcal{S} \cup \{p(i-1)\cdot x+r(i-1)\}$}\;
    {$\displaystyle\DP[i] \gets \min_{ax+b\in\mathcal{S}} \{a\cdot q(i)+b\}+s(i)$}\;
  }
\end{alg}

このような遷移に変形させることができれば勝ちなので,それを意識して式変形するとよい.
典型的には,以下のような遷移をするDPなどがこの枠組みに当てはまる.
\begin{align*}
  \DP[i]
  &= \min_{0\le j < i}\,\{\DP[j] + (i-j)^2 + C\}\\
  &= \min_{0\le j < i}\,\{\DP[j] + i^2-2ij+j^2+C\}\\
  &= \min_{0\le j < i}\,\{\underset{p(j)}{\underbrace{(-2j)}}\cdot \underset{q(i)}{\underbrace{\mathstrut i}} + \underset{r(j)}{\underbrace{(j^2+\DP[j])}}\} + \underset{s(i)}{\underbrace{(i^2+C)}}.
\end{align*}
convex hull trickの実装としてLi--Chao treeを用いる場合などは,$q(i)$の取りうる最大値が小さくなるように変形すると,クエリ先読みなどの必要がなくなることがあったりしてうれしそう.
もっとも,\texttt{std::map}などを利用してオンラインで処理することもできる.

\end{document}
