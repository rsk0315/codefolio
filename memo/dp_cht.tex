\documentclass{jsarticle}

\usepackage{rsk0315}

\newcommand{\DP}{\ensuremath\mathrm{dp}}

\begin{document}
\section{CHTで高速化するDP}

以下のような遷移をするDPを考える.
\begin{alg}
  \caption{愚直なDP}
  \For{$i \in \{1, \dots, N-1\}$}
  {
    {$\displaystyle\DP[i] \gets \min_{0\le j < i}\,\{p(j)\cdot q(i) + r(j)\} + s(i)$}\;
  }
\end{alg}

\noindent
$p(j)$や$r(j)$は$\DP[j]$を含む式でもいいし,関係ない式でも問題ない.

ここで,直線の集合に関する以下の処理をできるデータ構造を用意する.これはconvex hull trickなどと呼ばれるものである.
\begin{itemize}
\item 直線$y=ax+b$を追加する
\item 管理している直線のうち,$x=x_0$での$y$の最小値を答える
\end{itemize}

これを用いると,上のDPは以下のように高速化できる.
直線の集合を$\mathcal{S}$とし,このデータ構造を用いて管理する.
また,$\DP[0]$の値は計算できているとする.
\begin{alg}
  \caption{CHTで高速化したDP}
  {$\mathcal{S} \gets \{p(0)\cdot x+r(0)\}$}\;
  \For{$i \in \{1, \dots, N-1\}$}
  {
    {$\displaystyle\DP[i] \gets \min_{ax+b\in\mathcal{S}} \{a\cdot q(i)+b\}+s(i)$}\;
    {$\mathcal{S} \gets \mathcal{S} \cup \{p(i)\cdot x+r(i)\}$}\;
  }
\end{alg}



\end{document}
