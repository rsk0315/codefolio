\documentclass[dvipdfmx]{jsarticle}

\usepackage{rsk0315}
\newcommand{\DP}{\mathrm{dp}}
\newcommand{\floor}[1]{\lfloor #1\rfloor}

\title{Karatsuba法の概略}
\author{えびちゃん}

\begin{document}
\maketitle

多項式$A(x)$, $B(x)$が以下のように与えられる.
\begin{align*}
  A(x) &= a_0+a_1x+\dots+a_n x^n,\\
  B(x) &= b_0+b_1x+\dots+b_n x^n.
\end{align*}
このとき,その積$A(x)\cdot B(x)$を求めたい.
\[A(x)\cdot B(x) = a_0 b_0+(a_0 b_1 + a_1 b_0)\,x+\dots+a_n b_n x^{2n}.\]
%
ここで,まず$A(x)$を以下のように分解することを考える.
\begin{align*}
  A(x) &= a_0 + a_1 x + \dots + a_{\floor{n/2}} x^{\floor{n/2}}
  + a_{\floor{n/2}+1}\, x^{\floor{n/2}+1} + \dots + a_n x^n\\
  &= \underset{A_L(x)}{\underbrace{\left(a_0 + a_1 x + \dots + a_{\floor{n/2}} x^{\floor{n/2}}\right)}}
  + \underset{A_H(x)}{\underbrace{\left(a_{\floor{n/2}+1}\, x + \dots + a_n x^{n-\floor{n/2}}\right)}}\cdot x^{\floor{n/2}}\\
  &= A_L(x) + A_H(x)\cdot x^{\floor{n/2}}.
\end{align*}
$B(x)$についても同様に$B_L(x)$と$B_H(x)$に分解する.
%
これを用いると,所望の式は以下のように書き直せる.
\begin{align*}
  A(x)\cdot B(x)
  &= \left(A_L(x) + A_H(x)\cdot x^{\floor{n/2}}\right) \cdot \left(B_L(x) + B_H(x)\cdot x^{\floor{n/2}}\right)\\
  &= A_L(x)\cdot B_L(x)
  + \big(A_L(x)\cdot B_H(x)+A_H(x)\cdot B_L(x)\big)\cdot x^{\floor{n/2}}
  + A_H(x)\cdot B_H(x)\cdot x^{2\floor{n/2}}.
\end{align*}
$A\cdot B$を$A_L\cdot B_L$など半分のサイズの問題4つに分解できたことになるが,これでは得をしていない.
そこで,真ん中の項についてさらに変形をする.
\begin{align*}
  A_L(x)\cdot B_H(x) + A_H(x)\cdot B_L(x)
  &= \begin{aligned}[t]
    &\phantom{{}-{}}\big(A_L(x)+A_H(x)\big)\cdot\big(B_L(x)+B_H(x)\big)\\
    &- A_L(x)\cdot B_L(x) - A_H(x)\cdot B_H(x).
  \end{aligned}
\end{align*}
これにより,半分のサイズの解くべき問題が以下の3つになったことがわかる.
\begin{itemize}
\item $A_L(x)\cdot B_L(x)$
\item $A_H(x)\cdot B_H(x)$
\item $\big(A_L(x)+A_H(x)\big)\cdot\big(B_L(x)+B_H(x)\big)$
\end{itemize}
よって,分割統治のえらい定理から,$O(n^{\log_2 3})=O(n^{1.59})$時間で$A(x)\cdot B(x)$を計算できることがわかる.

\end{document}
