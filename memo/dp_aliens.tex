\documentclass{jsarticle}

\usepackage{rsk0315}

\newcommand{\floor}[1]{\left\lfloor #1\right\rfloor}
\newcommand{\ceil}[1]{\left\lceil #1\right\rceil}
\newcommand{\bitand}{\text{ \& }}

\title{WQS Binary Search}

\begin{document}
\maketitle

Trick from AliensとかAlien's DPとか\footnote{Aliens' DPだったりする?}.

関数にいい感じの性質を仮定する.

% Convex QI (QI for quadratic inequality) is なに

$k$回ちょうどなんかをしたときの最適値を求めたい.これは一般に難しい.

なんかをする際にペナルティ$\lambda$が発生するとする.

そのとき,いい感じの性質があると(まだちゃんと知らない),ペナルティで三分探索して操作を$k$回にするように調整できて,そこからペナルティぶんを差し引くことで$k$回行うときの最適化をする元問題を解ける.

回数$k$が離散なことに由来してる? それとも最適化される値による? ちゃんとわかってないけど,その性質から,ペナルティの候補は離散値に絞ることができて,それはつまり差分の二分探索でいいことになる.

ペナルティつき回数制約なし版の問題を$T(n)$で解けるとする.
また,その問題で最適解を達成する最小操作回数が$0$にできるようなペナルティの最小値(の上界)を$\lambda_{\mathrm{sup}}$とする.
このとき,$O(T(n)\log\lambda_{\mathrm{sup}})$で元問題が解ける.

\end{document}
